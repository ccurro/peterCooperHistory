\documentclass{article}
\usepackage{xcolor}
\usepackage{microtype}
\usepackage{ragged2e}
\usepackage{hanging}
\usepackage[paperwidth=8.5in,paperheight=11in,left=0.5in,right=0.5in,top=0.3in,bottom=0.5in]{geometry}
\usepackage{fontspec}
\setmainfont{Gibbs Book}[VerticalPosition=Ordinal,Ligatures={TeX,Common}]
\pagestyle{empty}
\setlength{\parindent}{0pt}


\usepackage{enumitem}

\newlist{enumList}{enumerate}{1}
\setlist[enumList]{label=\fontspec{Gibbs Bold}\arabic*.}

\renewcommand{\refname}{\scshape \fontsize{16pt}{16pt}\selectfont \fontspec
{Gibbs Bold} References}
\begin{document}
\centering
\fontspec{Gibbs Bold}[Letters=Uppercase]
\fontsize{0.8in}{0.6in}\selectfont
\scshape 
\addfontfeature{LetterSpace=0.0} Peter Cooper \\
\addfontfeature{Scale=0.85,LetterSpace=5} a Model Citizen

\vspace{0.05in}

\fontsize{16pt}{16pt}\selectfont
\fontspec{Gibbs Black}
\justify
Facts %:

\vspace{0.1in}

\begin{minipage}[t]{0.46\linewidth}
\fontspec{Gibbs Book}[VerticalPosition=Ordinal,Ligatures={Common,TeX}]
\fontsize{10.7pt}{12.7pt}\selectfont
\raggedright
\upshape
\begin{enumList}

\item Date of Birth: Feb. 12th, 1791 (Exactly 18 years prior to the birth of
{\fontspec{Gibbs Medium} Abraham Lincoln})

\item Date of Death: Apr. 4th, 1883 (14~years after the death of his wife
{\fontspec{Gibbs Medium} Sarah Cooper})

\item He received {\fontspec{Gibbs Medium} very little formal education}. In his
own words he said he had
only ``some three or four quarters'' at a public school and that ``there were no
night schools or laboratories nor any means by which an apprentice boy could get
information.''

\item His father was a hatter; on his work for his father he said ``I
remember\ldots being set to pull the hair out of rabbit skins, when {\fontspec
{Gibbs Medium} my head was just above the table}.''

\item On the risk he took at age 17: ``The only time I ever trusted to chance
for a profit\ldots I got a very wholesome lesson. I had earned about 10 dollars
beyond my immediate wants, which I invested, by the advice of a relative in
lottery tickets, all which, fortunately for me, drew blanks. This impressed
upon me the {\fontspec{Gibbs Medium} folly of looking to games of chance} for any
source of gain or livelihood.''

\item At age 17, he took on an apprenticeship as a coach-maker; he remained in
this capacity for four years until his employer offered to build him a shop and
setup a business for him -- a great offer, but in his own words, ``I always had a
{\fontspec{Gibbs Medium} horror of being burdened with debt}, and having no
capital of my own, {\fontspec{Gibbs Medium} I declined his kind offer}.''

\item His first profitable invention was a machine for shearing cloth. His first
sale of said machine was to Mr. Vassar, the founder of ``the noble institution
for female education, called the Vassar College.''

\item Soon afterwards he engaged in business as a grocer. Using the money from
the grocery and his sale of machines he purchased a {\fontspec{Gibbs Medium}
glue factory} between 31st
and 34th streets while {\fontspec{Gibbs Medium} incurring no debt}.
\newcounter{enumTemp}
\setcounter{enumTemp}{\value{enumListi}}
\end{enumList}
\end{minipage}\hfill
\begin{minipage}[t]{0.46\linewidth}
\fontspec{Gibbs Book}[VerticalPosition=Ordinal,Ligatures={Common,TeX}]
\fontsize{10.7pt}{12.7pt}\selectfont
\raggedright
\upshape
\begin{enumList}
\setcounter{enumListi}{\value{enumTemp}}

\item In 1828 he entered into a partnership with two fellows to purchase 3,000
acres of land in Baltimore for \$105,000. On this land he built the
{\fontspec{Gibbs Medium} Canton Iron Works}. When purchasing the land, he
discovered his partners had deceived him in their ability to pay so he bought
out their stakes and went at the venture alone.

\item Also in 1828 (at age 37) he was elected to his first public office, as
{\fontspec{Gibbs Medium} Assistant Alderman} of the City of New York. His first
order of business as a public officer was locating a clean source of water
for the city outside of its limits. His work culminated in the construction of
the {\fontspec{Gibbs Medium} Croton Aqueduct} and the {\fontspec{Gibbs Medium}
High Bridge}, the oldest standing bridge in the City of New York.

\item During the remainder of his time as an alderman, Peter Cooper participated in other important legislation such as those acts that {\fontspec{Gibbs Medium} allowed the formation of the FDNY and NYPD}.

\item Soon after purchasing the Baltimore land, he produced the
{\fontspec{Gibbs Medium} first steam
locomotive in the United States}. His locomotive Tom Thumb used his efficient
invention for ``rotary motion obtained for rectilinear alternating motion,
without a crank.'' Tom Thumb raced against a horse-drawn train,
to demonstrate its superior speed. While Tom Thumb broke down during the race,
its commanding lead up to that point proved its value.

\item In 1853 he purchased 19,000 acres in Ringwood, New Jersey, which included
the {\fontspec{Gibbs Medium} Long Pond Ironworks}, which was managed by
Edward Cooper and Abram S. Hewitt, and {\fontspec{Gibbs Medium} Ringwood Manor}
which now acts as a museum. Peter Cooper later installed {\fontspec{Gibbs
Medium} grand arched windows recycled from Cooper Union's Foundation Building}
at the manor.

\item He founded an ironworks at 33rd St. near 3rd~Ave. which became a
mill for rolling iron and making wire. Later this factory {\fontspec{Gibbs
Medium} rolled I-beams}, the first of their kind, to provide structure for
{\fontspec{Gibbs Medium} Cooper Union's Foundation Building}, making it largely
{\fontspec{Gibbs Medium} fire-proof}.

\setcounter{enumTemp}{\value{enumListi}}
\end{enumList}
\end{minipage}
\vspace*{\fill}

\fontspec{Gibbs Medium}
\hfill Over {\addfontfeature{Scale=0.8} →}
%
\newpage{}
%
\begin{minipage}[t]{0.46\linewidth}
\fontspec{Gibbs Book}[VerticalPosition=Ordinal,Ligatures={Common,TeX}]
\fontsize{11pt}{13pt}\selectfont
\raggedright
\upshape
\begin{enumList}
\setcounter{enumListi}{\value{enumTemp}}

\item He soon created several ironworks in New Jersey which he later
consolidated into the {\fontspec{Gibbs Medium} Trenton Ironworks}.

\item He also served as the longtime {\fontspec{Gibbs Medium} head of the Public School Society} which later became of the New York City Board of Education.

\item In 1857 the New York State Legislature passed an act to give ``corporate
powers to the Trustees of Peter Cooper's Munificent Gift to Science and Art.''
This is the first public record of Cooper Union, and in this act Cooper Union is
declared a {\fontspec{Gibbs Medium} public institution with a board of
trustees composed of representatives of other public institutions in the City and
State of New York}, such as the Governor's office, the New-York Historical
Society, and the Astor Library.

\item In 1859, Peter Cooper opened The Cooper Union, a {\fontspec{Gibbs Medium}
free school that admitted all students on the basis of merit alone.}
\item In 1860, {\fontspec{Gibbs Medium} Abraham Lincoln} delivered his
{\fontspec{Gibbs Medium}[Ligatures=TeX] ``Right Makes Might''} speech at Cooper
Union.
Historians contend that this speech catapulted him to the presidency, and that
{\fontspec{Gibbs Medium} without Cooper Union, the United States may soon have
fallen.}

\item Peter Cooper, with Cyrus Field and others, financed and
pioneered the laying of the {\fontspec{Gibbs Medium} first Transatlantic
telegraph cable}. Despite several majors setbacks Peter Cooper persevered and
ultimately it was due to his spirit that the project succeeded.

\item In 1876, he was {\fontspec{Gibbs Medium} nominated to run for President of
the United States} by the Greenback party; Peter Cooper is still the
{\fontspec{Gibbs Medium} oldest person} nominated for the presidency by a political party.

\item Throughout his later life, Peter Cooper often penned open letters to the
legislature and other offices of government, including the presidency. Many of
these letters were compiled by J.D. Zachos in 1877 \cite{opinions}.

\item After his death in the 1883, the New York Times described the
{\fontspec{Gibbs Medium} thousands of people that paid their respects} both at
All Souls' Church (estimated at 15,000 people) and an even larger number of
people that lined the streets along the procession route.

\item Peter Cooper is buried, with other members of the Cooper-Hewitt family,
at {\fontspec{Gibbs Medium} Green-Wood Cemetery} in Brooklyn. (Open to vistors)

% He has an isolated plot with the other immediate members of the
% Cooper-Hewitt family.

\setcounter{enumTemp}{\value{enumListi}}
\end{enumList}
\end{minipage}\hfill
\begin{minipage}[t]{0.46\linewidth}
\fontspec{Gibbs Book}[VerticalPosition=Ordinal,Ligatures={TeX,Common}]
\fontsize{11pt}{13pt}\selectfont
\raggedright
\upshape
\begin{enumList}
\setcounter{enumListi}{\value{enumTemp}}

\item Miscellaneous Family Details
\begin{itemize}[leftmargin=*]
\item Edward Cooper, Son, Mayor of New York, President of Cooper Union
\item Abram S. Hewitt, Son-in-law, Mayor of New York, President of Cooper Union,
Financed first subway line in New York
\item Sarah Cooper Hewitt, Eleanor Garnier Hewitt and Amy Hewitt Green,
Granddaughters, founded the Cooper-Hewitt Museum
\end{itemize}

\item Other Miscellaneous Inventions
\begin{itemize}[leftmargin=*]
\item A method for extracting power from water currents that reacted both to the
tide and speed of the current to garner greater efficiency. (Created while
still a coach-maker's apprentice)
\item A horseless method of towing along the Erie Canal -- not used because, to gain the right to construct the canal, Governor Clinton promised
farmers along the proposed canal path that the horses pulling the boats would be fed by their grain.
\item A conveyor belt system for transporting iron ore powered
primarily by gravity; used in his various ironworks operations.
\item A self rocking cradle, with music-box used to bring his children to sleep.
\item The powdered gelatin now known as Jell-O.
\end{itemize}
\end{enumList}

% \vspace{0.05in}

\fontspec{Gibbs Book}[Numbers=Monospaced,Ligatures={Common,TeX}]
\fontsize{9.5pt}{9.5pt}\selectfont
\upshape
\raggedright
\nocite{*}
\bibliographystyle{IEEEtran}
\bibliography{bib}

\vspace{0.1in}

\scshape \fontsize{16pt}{16pt}\selectfont \fontspec{Gibbs Black} About
\fontspec{Gibbs Book}[Numbers=Monospaced]
\fontsize{10pt}{10pt}\selectfont
\upshape
\raggedright

\begin{description}
\itemsep1pt
\item[Contributors:] Chris Curro, {\scshape ee '15}; 
\item[Typefaces:] Gibbs in Book, Medium, Bold and Black;
\item[Special Thanks:] Amanda Machnik, {\scshape a '15};
\item[Printing:] First Edition, August 2015
\end{description}
\end{minipage}
\end{document}
